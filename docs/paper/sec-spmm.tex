% === Section III: Two-Relation Join = SpMM ===
\section{Two-Relation Join Recovers SpMM}
\label{sec:spmm}

Before tackling the triangle query, we demonstrate a remarkable
correspondence: the two-relation join $R(x,y) \bowtie S(y,z)$ is
precisely sparse matrix--matrix multiplication (SpMM),
$C = R \times S$.

Evaluating the join variable at a time with ordering $(x,y,z)$:

\smallskip
\begin{center}
\small
\begin{tabular}{@{}clll@{}}
\toprule
Level & Var.\ & Constraints & Candidates \\
\midrule
1 & $x$ & $R(x,\cdot)$ & rows of $R$: $\{1, \ldots, N\}$ \\
2 & $y$ & $R(x,y),\; S(y,\cdot)$ & $N_R(x) \cap \text{rows}(S)$ \\
3 & $z$ & $S(y,z)$ & $N_S(y)$ \\
\bottomrule
\end{tabular}
\end{center}
\smallskip

\noindent
At each level, the variable is constrained by the relations containing
it.  Crucially, no level has more than one relation contributing
\emph{values}---level~2's intersection is trivial when $S$ covers all
row indices $1{:}N$ (since $N_R(x) \cap \{1, \ldots, N\} = N_R(x)$).
The join reduces to:
%
\begin{align*}
&\textbf{for } x = 1 \textbf{ to } N\textbf{:}\\
&\quad \textbf{for } y \in N_R(x)\textbf{:}\\
&\quad\quad \textbf{for } z \in N_S(y)\textbf{:}\\
&\quad\quad\quad C[x,z] \mathrel{+}= R[x,y] \cdot S[y,z]
\end{align*}
%
This is the textbook \emph{row-by-row SpMM} algorithm.
No intersections are needed---each level's variable is constrained by
at most one relation, so the evaluation is pure nested iteration.

\paragraph{Connection to GraphBLAS.}
The GraphBLAS \texttt{GrB\_mxm} operation computes this two-relation
join.  GraphBLAS triangle counting~\cite{aznaveh2020,wolf2017}
extends it to three relations via a mask: $C\langle L \rangle = L
\cdot L$, which first computes the two-relation SpMM and then filters
by the third relation.  This is exactly the pairwise strategy
that Section~\ref{sec:triangle} analyzes.

The key observation: SpMM is the special case where each variable is
constrained by at most one relation, requiring no intersection.
Triangle counting---the three-relation case---is where multiple
relations constrain the same variable, and intersection becomes
essential.  The fused approach intersects where the pairwise approach
decomposes.
