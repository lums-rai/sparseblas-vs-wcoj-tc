% === Section VII: Why Lazy Evaluation Cannot Close the Gap ===
\section{Why Lazy Evaluation Cannot Close the Gap}
\label{sec:lazy}

A natural objection to the preceding analysis is that the pairwise
blowup is an implementation artifact: if the SpMM were evaluated
\emph{lazily}---fusing the multiply and mask phases without
materializing the intermediate---the $\Oh(M^2)$ cost would
disappear.  This section explains why lazy evaluation within the
linear algebra framework is fundamentally insufficient.

\subsection{The Structural Limitation}

The pairwise SpMM approach computes $C\langle L \rangle = L \cdot L$
in two conceptual phases: (1)~for each pair $(i,k)$, sum over shared
indices $j$ to form $C_{ik}$, and (2)~discard entries where
$L_{ik} = 0$.  Lazy evaluation can fuse these phases, checking the
mask before accumulating.  But the fused pairwise computation still
enumerates the \emph{same} set of $(i,j,k)$ triples---it merely
avoids storing the non-triangles.  The \emph{work} is unchanged:
every length-2 path $(i \to j \to k)$ is visited, regardless of
whether the closing edge $(i,k)$ exists.

The WCOJ approach operates differently.  At the innermost level, it
intersects $N(x) \cap N(y)$---only visiting $z$-values that satisfy
\emph{both} constraints simultaneously.  Length-2 paths that do not
close into triangles are never explored, because the intersection
prunes them before they are enumerated.

Lazy evaluation of the pairwise plan eliminates the
\emph{materialization} cost but not the \emph{enumeration} cost.
The WCOJ plan eliminates both.

\subsection{A Formal Basis: The SPORES Result}

Wang et al.~\cite{wang2020spores} formalized this limitation.  Their
SPORES system optimizes linear algebra (LA) expressions by lifting
them into relational algebra (RA), applying RA equivalence rules, and
lowering the result back to LA\@.  The central theorem is:

\begin{quote}
\emph{RA rewrite rules are complete for LA optimization}: any
equivalent LA expression can be reached via RA rewrites.  The converse
does not hold---LA rewrite rules alone cannot reach all equivalent LA
expressions.
\end{quote}

The reason is structural.  LA expressions are restricted to operations
on matrices (two-index objects).  RA can introduce intermediate
expressions with \emph{three or more free variables}---higher-arity
relations that have no matrix representation.  These higher-arity
intermediates serve as ``stepping stones'' to reach LA expressions that
are unreachable by pairwise matrix rewrites alone.

For triangle counting, the fused three-way intersection is precisely
such a higher-arity operation: it simultaneously binds $x$, $y$,
and~$z$ across three relations.  No sequence of pairwise matrix
operations---however cleverly fused or lazily evaluated---can express
this simultaneous three-way constraint.  The limitation is not in the
implementation but in the \emph{algebra}: pairwise operations are
simply less expressive than multiway operations for cyclic patterns.

\subsection{Implications}

This has a concrete consequence for the sparse linear algebra
community.  Efforts to improve GraphBLAS triangle counting through
better SpMM implementations, smarter masking strategies, or lazy
evaluation of expression trees are optimizing within an algebraic
framework that is provably incomplete.  For triangle counting (and
more generally, for any cyclic join pattern), the optimal algorithm
lives outside the space of pairwise matrix expressions.

The relational algebra framework reaches it naturally.  And as
Sections~\ref{sec:equivalence}--\ref{sec:production} demonstrate, the
resulting code is not some exotic database construction---it is the
same CSR intersection loop that HPC programmers already write.
