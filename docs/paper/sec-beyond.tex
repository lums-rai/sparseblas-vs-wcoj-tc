% === Section VIII: Beyond Triangles ===
\section{Beyond Triangles: Cyclic Patterns in General}
\label{sec:beyond}

Triangle counting is the simplest cyclic graph query, but the gap
between pairwise and fused evaluation widens for larger patterns.
This section applies the WCOJ construction of
Section~\ref{sec:construction} to additional cyclic queries, showing
that the masked SpGEMM technique---which partially closes the gap for
triangles---does not generalize.

\subsection{Why Masked SpGEMM Is Triangle-Specific}
\label{sec:masked-specific}

Recall the GraphBLAS triangle-counting formulation:
$C\langle L \rangle = L \cdot L$.  SuiteSparse implements this via
\emph{dot-product masked SpGEMM}: for each edge $(i,j)$ in the
mask~$L$, compute the inner product of row~$i$ and column~$j$ of~$L$.
This effectively intersects $N(i) \cap N(j)$ for each edge---the same
work that the WCOJ fused kernel performs.

This implementation succeeds because of three properties specific to
triangles:
\begin{enumerate}
\item The mask is the \emph{same} edge set $L$ being multiplied---known
  in advance, with $M$ entries.
\item The output is indexed by edges (sparse), not by arbitrary vertex
  pairs (potentially dense).
\item A single masked inner product per edge suffices to fuse the
  three-way check.
\end{enumerate}

\noindent
For other cyclic patterns, at least one of these properties fails.

\subsection{4-Cycle (Square) Counting}
\label{sec:4cycle}

The 4-cycle query asks for all tuples $(a,b,c,d)$ such that edges
$(a,b)$, $(b,c)$, $(c,d)$, and $(a,d)$ all exist:
\[
  Q_4(a,b,c,d) \;=\; R(a,b) \;\wedge\; R(b,c) \;\wedge\; R(c,d)
    \;\wedge\; R(a,d).
\]

\paragraph{The linear algebra approach.}
The standard formulation computes the 2-path matrix $P = A^2$, where
$P[a,c]$ counts the number of vertices~$b$ such that $(a,b)$ and
$(b,c)$ are edges.  Each pair of distinct 2-paths between the same
endpoints forms a 4-cycle, so the count is $\sum_{a < c} \binom{P[a,c]}{2}$.

The problem: \textbf{there is no natural mask.}  The computation requires
$P[a,c]$ for every pair $(a,c)$ with two or more common neighbors.
Which pairs are those?  Unknown until $P$ is computed.  And $P$ can be
\emph{dense}: a star graph with hub degree~$d$ produces $\binom{d}{2}$
nonzero entries in $P$, even though $A$ has only $d$ edges.  The
``mask'' would be $\{(a,c) : P[a,c] \ge 2\}$---potentially $\Oh(n^2)$
entries, exactly the blowup that masking was supposed to prevent.

\paragraph{The WCOJ approach.}
Applying the construction of Section~\ref{sec:construction} with
variable ordering $(a, b, d, c)$:

\begin{center}
\small
\begin{tabular}{@{}cllll@{}}
\toprule
Depth & Var.\ & Atoms & Candidates & Action \\
\midrule
0 & $a$ & $R(a,b),\; R(a,d)$ & vertices & iterate \\
1 & $b$ & $R(a,b)$ & $N(a)$ & iterate \\
2 & $d$ & $R(a,d)$ & $N(a),\; d \neq b$ & iterate \\
3 & $c$ & $R(b,c),\; R(c,d)$ & $\mathbf{N(b) \cap N(d)}$ & \textbf{intersect} \\
\bottomrule
\end{tabular}
\end{center}

At depth~3, WCOJ intersects $N(b) \cap N(d)$---only $c$-values that
close the 4-cycle survive.  Triples $(a,b,d)$ where $b$ and $d$ share
no common neighbor are pruned immediately.  The pairwise approach
computes all of $A^2$ first, then sifts through it.

\paragraph{Complexity.}
The AGM bound for the 4-cycle with $M$-edge relations is
$\Oh(M^{3/2})$; the best known WCOJ-style algorithm (PANDA~\cite{abokhamis2016faq})
achieves $\Oh(M^{3/2})$.  Any pairwise decomposition
requires $\Oh(M^2)$ in the worst case, because computing the full
2-path matrix is unavoidable.

\subsection{Other Cyclic Patterns}

The same analysis applies to every cyclic subgraph pattern:

\begin{description}
\item[Diamond ($K_4$ minus one edge).]
  Four vertices, five edges.  The query has two cycles.
  Pairwise: $\Oh(M^2)$.  WCOJ: $\Oh(M^{3/2})$.

\item[Bow-tie (two triangles sharing a vertex).]
  Five variables, six constraints, two triangular cycles joined at a
  hub vertex.  The pairwise approach requires computing per-vertex
  triangle counts and then combining---multiple stages, compounding
  intermediates.  WCOJ processes all five variables with intersections
  at the cyclic depths, fusing both triangles into a single pass.

\item[$k$-cycle ($C_k$ for $k \ge 3$).]
  The AGM bound gives $\Oh(M^{k/2})$; pairwise decomposition is stuck
  at $\Oh(M^2)$ because any tree decomposition of a $k$-cycle has
  width $\lceil k/2 \rceil$.  The gap grows with $k$.

\item[$k$-clique ($K_k$).]
  The AGM bound is $\Oh(M^{k/2})$, which pairwise plans also achieve,
  so the worst-case gap is in constant factors.  However, the WCOJ
  approach still avoids intermediate materialization and benefits from
  intersection pruning on real-world data.
\end{description}

\subsection{The General Criterion}

The dividing line is the \textbf{cyclicity of the query hypergraph}.
Grohe and Marx~\cite{grohe2014} established:

\begin{itemize}
\item \emph{Acyclic queries} (fractional hypertree width $= 1$):
  Yannakakis's algorithm computes the join in $\Oh(N + |\text{output}|)$
  time using semijoins.  Binary join plans suffice.  WCOJ provides no
  asymptotic benefit.

\item \emph{Cyclic queries} (fractional hypertree width $> 1$):
  Any binary join-project plan is polynomially slower than the AGM
  bound in the worst case.
\end{itemize}

\noindent
Ngo et al.~\cite{ngo2013skew} proved the stronger statement: for
cyclic queries, ``any join-project plan is destined to be slower than
the best possible run time by a polynomial factor in the data size.''

In graph terms: acyclic queries are path/tree pattern matching
(two-way joins suffice).  Every query involving a \emph{cycle}---triangles,
squares, diamonds, cliques, bow-ties, wheels---requires multiway
intersection for optimal evaluation.  Since sparse linear algebra
operates via pairwise matrix products, it is structurally limited to
binary join plans.  The masked SpGEMM trick recovers optimal
performance for triangles as a special case, but the general cyclic
case requires the WCOJ framework.

Table~\ref{tab:cyclic-patterns} summarizes the complexity landscape.

\begin{table}[t]
\centering
\caption{Pairwise vs.\ WCOJ complexity for cyclic graph patterns.
  $M$ = number of edges in each relation.}
\label{tab:cyclic-patterns}
\small
\begin{tabular}{@{}llccc@{}}
\toprule
Pattern & Query & Pairwise & WCOJ & Gap \\
\midrule
2-path (acyclic) & $R \bowtie S$ & $\Oh(M)$ & $\Oh(M)$ & --- \\
Triangle ($C_3$) & 3-way & $\Oh(M^2)$ & $\Oh(M^{3/2})$ & $M^{1/2}$ \\
4-cycle ($C_4$) & 4-way & $\Oh(M^2)$ & $\Oh(M^{3/2})$ & $M^{1/2}$ \\
Diamond ($K_4 - e$) & 5-way & $\Oh(M^2)$ & $\Oh(M^{3/2})$ & $M^{1/2}$ \\
$k$-cycle ($C_k$) & $k$-way & $\Oh(M^2)$ & $\Oh(M^{2-\frac{1}{\lceil k/2 \rceil}})$ & polynomial \\
$k$-clique ($K_k$) & $\binom{k}{2}$-way & $\Oh(M^{k/2})$ & $\Oh(M^{k/2})$ & const. \\
\bottomrule
\end{tabular}
\end{table}
